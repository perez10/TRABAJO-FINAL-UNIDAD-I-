\documentclass{article}
\usepackage[utf8]{inputenc}

\begin{document}


\caption\textbf{CONCLUSIONES}
\\
\begin{itemize}
\\\item En forma exclusiva, la inteligencia empresarial y el análisis de negocios forman los componentes esenciales requeridos por una empresa para administrar su información de manera efectiva.
\\
\item 
Las dos terminologías parecen tener similitudes de una manera que hace que los estudiantes deduzcan que están conectados. De hecho, vale la pena reconocer que la analítica es una función de la inteligencia empresarial. La información analizada mediante análisis para predecir el futuro es la misma información obtenida del componente de inteligencia de la empresa. Tanto BI como BA se enfocan en impulsar a la compañía a avanzar enérgicamente
\\
\item En un medio globalizado y audaz como el del mundo empresarial, podemos ver que el entorno en el que la inmensa mayoría de las empresas tiene soportados los procesos de negocio con diferentes sistemas de información y estrategias, los ubica en un mercado tan competitivo como el actual, Hoy se ha convertido en un problema, por lo que la Inteligencia de Negocios se erige como una pieza clave para ser proactivo a la hora de tomar mejores decisiones y de conseguir mejor control de negocio y ventajas que nos diferencien de la competencia.
\\
\item Lo que HEMOS APRENDIDO es que la gran mayoría de empresas no utilizan sistemas de inteligencia empresarial para gestionar sus negocios. Sin embargo, SABEMOS QUE SI ENTIENDEN EL CONCEPTO, y saben que son herramientas muy enriquecedoras para la gestión actual, a lo que añaden las siguientes ventajas para el  uso de Software de Inteligencia de Negocios:

 – Los sistemas de Business Intelligence ayudan a hacer más competitiva la estrategia de la empresa.

 – Apoyan la toma de decisiones que son vitales para obtener mejores resultados

 – La Inteligencia de Negocios facilita notablemente la interactividad entre usuarios, clientes y proveedores.

 – Facilitan el acceso a los datos críticos de la empresa y las informaciones corporativas para la integración de datos y la toma de decisiones

 – Permite alinear acciones de diferentes departamentos e igualmente ayuda a controlar cada línea de negocio o departamento con métricas específicas.
 \\
\end{itemize} 

\end{document}
